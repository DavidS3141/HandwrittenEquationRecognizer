\documentclass[twocolumn]{article}%
\usepackage[utf8]{inputenc}
\usepackage[T1]{fontenc}
\usepackage{geometry}
\usepackage{cuted}
\usepackage{lipsum}
\title{Handwritten Equation Recognizer (HER)}
\author{William Jussiau, Johnson Loh, David Schmidt}
\date{\today}
\setlength\columnsep{3em}

\begin{document}


%\begin{strip}
%  \vspace*{\dimexpr-\baselineskip-\stripsep\relax}
%  \centering
%  \maketitle
%  \vskip\baselineskip
%%\noindent\makebox[\textwidth]{\rule{1.1\paperwidth}{0.4pt}}
%  \vskip\baselineskip
%\end{strip}

\twocolumn[
  \begin{@twocolumnfalse}
    \maketitle
    \begin{abstract}
     	    The Handwritten Equation Recognizer (HER) is a novel approach to apply machine learning techniques in the context of pattern recognition. The focus of this work is to identify handwritten equations based on image data. The identification procedure consists of two steps: cluster the pixels in the image to symbols and then classify the recognized clusters as mathematical symbols.
    \end{abstract}
  \end{@twocolumnfalse}
]

 

	
	\section{Introduction}
		The field of handwriting recognition is an interesting application task for advanced machine learning algorithms. In tournaments like the Competition on Recognition of Handwritten Mathematical Expressions (CROHME) \cite{crohme}, different teams compare their state of the art algorithms on a huge dataset. Usually, this dataset consists of stroke sequences, which describe the generation of a whole mathematical formula.\\
		In contrast to using this sort of data (i.e. stroke sequence), our approach is using the image of the mathematical expression. The information of how a symbol is actually constructed is not used. Instead, the image is preprocessed into a binary image using the an optimal threshold. Then, based upon the pixel occupancy and their location, two steps of machine learning are applied. The first step is a clustering algorithm that enables us to get the location and extent of each symbol in the formula. With this information, we can then apply the second step on each cluster of pixels: classification. Each symbol extracted by the clustering step is to be recognized as a known mathematical glyph using a classifier.  Hence, it is possible to reconstruct the handwritten formula.
	    	    
	    
	\section{Proposed method}
	\subsection{Clustering}
		The first algorithm to be applied aims at splitting the mathematical formula into single symbols. The method used in this project for identifying the symbols is the hierarchical agglomerative clustering algorithm. Using the functions provided in the Statistics and Machine Learning Toolbox from Matlab \cite{sml_matlab} the implementation is straight forward. The whole image is converted into a data array containing the two-dimensional coordinates of the occupied pixels in the image. Then the cluster tree is built using the Euclidean distance and single linkage as the cluster distance. Based on the generated dendrogram, a specified number of clusters can be derived. The amount of clusters can either be manually specified by the user or it can be generated by a cutoff distance $d_{cutoff}$. In this case, it makes sense to define this cutoff threshold as:
		\begin{equation}
		d_{symbol} > d_{cutoff} > \sqrt{\Delta x^2 + \Delta y^2}
		\end{equation}
		$\Delta x$ and $\Delta y$ denote respectively the horizontal and vertical pixel spacing. $d_{symbol}$ denotes the space between symbols. To provide an output for the following processing steps the bounding box of each recognized symbol is returned.
		
			The advantage of the hierarchical agglomerative clustering compared to other clustering methods is that it does not depend on the initialization of the clusters centres. As the assumption of the proposed method is that a symbol is defined as a connected space in the image space, the single linkage method is the most suited way to identify these symbols. The difficulties regarding symbols that are not connected in space (e.g. equal sign) can be solved by weighting vertical pixel space less than horizontal pixel space ($\Delta x > \Delta y$). 
		However, an obvious disadvantage is that connected symbols in some handwritings may be recognized as one single symbol. Another problem is to automatically define the number of symbols inside a given equation. Identifying the right point in the dendrogram is not trivial, as the horizontal pixel distance between the connection of two symbols and inside the symbol itself is the same.
	

	\subsection{Classification}
	
	
	\section{Experiment}
		
	
	\section{Conclusion}
	
	\begin{thebibliography}{9}
		\bibitem{crohme} 
		International Conference on Frontiers in Handwriting Recognition in Shenzen, China (ICFHR),
		\\\texttt{http://ivc.univ-nantes.fr/CROHME/}
		
		\bibitem{sml_matlab} 
		Statistics and Machine Learning Toolbox, The MathWorks, Inc., 1994-2017,
		\\\texttt{https://au.mathworks.com/products/statistics.html}
		
		
	\end{thebibliography}

\end{document} 