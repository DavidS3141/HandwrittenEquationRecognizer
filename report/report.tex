\documentclass[twocolumn]{article}%
\usepackage[utf8]{inputenc}
\usepackage[T1]{fontenc}
\usepackage{geometry}
\usepackage{cuted}
\usepackage{lipsum}
\title{Handwritten Equation Recognizer (HER)}
\author{William Jussiau, Johnson Loh, David Schmidt}
\date{\today}
\setlength\columnsep{3em}

\begin{document}

\begin{strip}
  \vspace*{\dimexpr-\baselineskip-\stripsep\relax}
  \centering
  \maketitle
  \vskip\baselineskip
%\noindent\makebox[\textwidth]{\rule{1.1\paperwidth}{0.4pt}}
  \vskip\baselineskip
\end{strip}


    \begin{abstract}
	    The Handwritten Equation Recognizer (HER) is a novel aproach to apply machine learning techniques in the context of pattern recognition. The focus of this work is to identify handwritten equations based on image data. The identification procedure consists of two steps: Cluster the pixels in the image to symbols and then classify the recognized clusters as mathematical symbols.
	\end{abstract}
	
	\section{Introduction}
		The field of handwriting recognition is a interesting application task for advanced machine learning algorithms. In tournaments like the Competition on Recognition of Handwritten Mathematical Expressions (CROHME) \cite{crohme} different teams compare their state of the art algorithms on a huge dataset. Usually, this dataset consists of stroke sequences, which describe the generation of a formular completely.\\
		In contrast to using the data directly, our approach is using the image of the mathematical expression. So the information of how a symbol is actually constructed is not used. Instead, the image is preprocessed into a binary image using the optimal threshold calculated Otsu's method. Then based upon the pixel occupancy and its location the machine learning algorithms are applied.
	    
	\section{Background}
	    
	    
	\section{Proposed method}
		The method used in this project for identifiing the symbols is the hierarchical agglomerative clustering algorithm. Using the functions provided in the Statistics and Machine Learning Toolbox from Matlab \cite{sml_matlab} the implementation is straight forward: the image is converted into a data array containing the 2D coordinates of the occupied pixels in the image. Then the cluster tree is build using the euclidean distance and single linkage as the cluster distance. Based on the generated dendrogram a specified number of clusters can be derived. The amount of clusters can either be manually specified by the user or it can be generated by a cutoff distance $d_{cutoff}$. In this case it makes sense to define this distance as
		\begin{equation}
		d_{symbol} > d_{cutoff} > \sqrt{\triangle x^2 + \triangle y^2}
		\end{equation}
		, while $\triangle x$, $\triangle y$ denotes the horizontal/vertical pixel spacing and $d_{symbol}$ denotes the space between symbols. To provide an output for the following processing steps the bounding box of each recognized symbol is returned.
		
	
	\section{Analysis}
		The advantage of the hierarchical agglomerative clustering compared to other clustering methods is that it is independent of the initialization. As the assumption of the proposed method is that a symbol is defined as a connected space in the image space, the single linkage method is the most suited way to identify these symbols. The difficulties regarding symbols that are not connected in space (e.g. equal sign) can be solved by weighting vertical pixel space less than horizontal pixel space ($\triangle x > \triangle y$). 
		So an obvious disadvantage is, that connected symbols in some handwritings are recognized as one single symbol. Another problem is to define the number of symbols inside a picture. Identifiing the right point in the dendrogram is not trivial, as the horizontal pixel distance between the connection of two symbols and inside the symbol itself is the same.
	
	\section{Experiment}
		
	
	\section{Conclusion}
	
	\begin{thebibliography}{9}
		\bibitem{crohme} 
		International Conference on Frontiers in Handwriting Recognition in Shenzen, China (ICFHR),
		\\\texttt{http://ivc.univ-nantes.fr/CROHME/}
		
		\bibitem{sml_matlab} 
		Statistics and Machine Learning Toolbox, The MathWorks, Inc., 1994-2017,
		\\\texttt{https://au.mathworks.com/products/statistics.html}
		
		
	\end{thebibliography}

\end{document} 